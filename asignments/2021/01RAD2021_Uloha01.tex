% Options for packages loaded elsewhere
\PassOptionsToPackage{unicode}{hyperref}
\PassOptionsToPackage{hyphens}{url}
%
\documentclass[
]{article}
\usepackage{lmodern}
\usepackage{amssymb,amsmath}
\usepackage{ifxetex,ifluatex}
\ifnum 0\ifxetex 1\fi\ifluatex 1\fi=0 % if pdftex
  \usepackage[T1]{fontenc}
  \usepackage[utf8]{inputenc}
  \usepackage{textcomp} % provide euro and other symbols
\else % if luatex or xetex
  \usepackage{unicode-math}
  \defaultfontfeatures{Scale=MatchLowercase}
  \defaultfontfeatures[\rmfamily]{Ligatures=TeX,Scale=1}
\fi
% Use upquote if available, for straight quotes in verbatim environments
\IfFileExists{upquote.sty}{\usepackage{upquote}}{}
\IfFileExists{microtype.sty}{% use microtype if available
  \usepackage[]{microtype}
  \UseMicrotypeSet[protrusion]{basicmath} % disable protrusion for tt fonts
}{}
\makeatletter
\@ifundefined{KOMAClassName}{% if non-KOMA class
  \IfFileExists{parskip.sty}{%
    \usepackage{parskip}
  }{% else
    \setlength{\parindent}{0pt}
    \setlength{\parskip}{6pt plus 2pt minus 1pt}}
}{% if KOMA class
  \KOMAoptions{parskip=half}}
\makeatother
\usepackage{xcolor}
\IfFileExists{xurl.sty}{\usepackage{xurl}}{} % add URL line breaks if available
\IfFileExists{bookmark.sty}{\usepackage{bookmark}}{\usepackage{hyperref}}
\hypersetup{
  pdftitle={1. zápočtová úloha z 01RAD},
  pdfauthor={Jiří Franc},
  hidelinks,
  pdfcreator={LaTeX via pandoc}}
\urlstyle{same} % disable monospaced font for URLs
\usepackage[margin=1in]{geometry}
\usepackage{longtable,booktabs}
% Correct order of tables after \paragraph or \subparagraph
\usepackage{etoolbox}
\makeatletter
\patchcmd\longtable{\par}{\if@noskipsec\mbox{}\fi\par}{}{}
\makeatother
% Allow footnotes in longtable head/foot
\IfFileExists{footnotehyper.sty}{\usepackage{footnotehyper}}{\usepackage{footnote}}
\makesavenoteenv{longtable}
\usepackage{graphicx,grffile}
\makeatletter
\def\maxwidth{\ifdim\Gin@nat@width>\linewidth\linewidth\else\Gin@nat@width\fi}
\def\maxheight{\ifdim\Gin@nat@height>\textheight\textheight\else\Gin@nat@height\fi}
\makeatother
% Scale images if necessary, so that they will not overflow the page
% margins by default, and it is still possible to overwrite the defaults
% using explicit options in \includegraphics[width, height, ...]{}
\setkeys{Gin}{width=\maxwidth,height=\maxheight,keepaspectratio}
% Set default figure placement to htbp
\makeatletter
\def\fps@figure{htbp}
\makeatother
\setlength{\emergencystretch}{3em} % prevent overfull lines
\providecommand{\tightlist}{%
  \setlength{\itemsep}{0pt}\setlength{\parskip}{0pt}}
\setcounter{secnumdepth}{-\maxdimen} % remove section numbering

\title{1. zápočtová úloha z 01RAD}
\author{Jiří Franc}
\date{2021-10-21}

\begin{document}
\maketitle

\hypertarget{zuxe1poux10dtovuxe1-uxfaloha-z-01rad}{%
\section{1. zápočtová úloha z
01RAD}\label{zuxe1poux10dtovuxe1-uxfaloha-z-01rad}}

\hypertarget{popis-uxfalohy}{%
\subsection{Popis úlohy}\label{popis-uxfalohy}}

V tomto úkolu je cílem provést předzpracování datového souboru, jeho
vizualizaci a jednoduchou lineární regresní úlohu, kde budeme modelovat
spotřebu automobilu v závislosti na jeho váze. K tomuto účelu poslouží
datový souboru \texttt{auto\_mpg\_2021rad.txt}, který obsahuje 406
pozorování o 9 proměnných. Dataset byl prvně použit americkou
statistickou společností v roce 1983 a lze ho též najít na UCI Machine
Learning Repository, případně na kaggle.com s několika pracovními
sešity.

\begin{longtable}[]{@{}rrrrrrrrl@{}}
\toprule
\begin{minipage}[b]{0.03\columnwidth}\raggedleft
mpg\strut
\end{minipage} & \begin{minipage}[b]{0.07\columnwidth}\raggedleft
cylinders\strut
\end{minipage} & \begin{minipage}[b]{0.10\columnwidth}\raggedleft
displacement\strut
\end{minipage} & \begin{minipage}[b]{0.08\columnwidth}\raggedleft
horsepower\strut
\end{minipage} & \begin{minipage}[b]{0.05\columnwidth}\raggedleft
weight\strut
\end{minipage} & \begin{minipage}[b]{0.10\columnwidth}\raggedleft
acceleration\strut
\end{minipage} & \begin{minipage}[b]{0.08\columnwidth}\raggedleft
model\_year\strut
\end{minipage} & \begin{minipage}[b]{0.05\columnwidth}\raggedleft
origin\strut
\end{minipage} & \begin{minipage}[b]{0.19\columnwidth}\raggedright
car\_name\strut
\end{minipage}\tabularnewline
\midrule
\endhead
\begin{minipage}[t]{0.03\columnwidth}\raggedleft
18\strut
\end{minipage} & \begin{minipage}[t]{0.07\columnwidth}\raggedleft
8\strut
\end{minipage} & \begin{minipage}[t]{0.10\columnwidth}\raggedleft
307\strut
\end{minipage} & \begin{minipage}[t]{0.08\columnwidth}\raggedleft
130\strut
\end{minipage} & \begin{minipage}[t]{0.05\columnwidth}\raggedleft
3504\strut
\end{minipage} & \begin{minipage}[t]{0.10\columnwidth}\raggedleft
12.0\strut
\end{minipage} & \begin{minipage}[t]{0.08\columnwidth}\raggedleft
70\strut
\end{minipage} & \begin{minipage}[t]{0.05\columnwidth}\raggedleft
1\strut
\end{minipage} & \begin{minipage}[t]{0.19\columnwidth}\raggedright
chevrolet chevelle malibu\strut
\end{minipage}\tabularnewline
\begin{minipage}[t]{0.03\columnwidth}\raggedleft
15\strut
\end{minipage} & \begin{minipage}[t]{0.07\columnwidth}\raggedleft
8\strut
\end{minipage} & \begin{minipage}[t]{0.10\columnwidth}\raggedleft
350\strut
\end{minipage} & \begin{minipage}[t]{0.08\columnwidth}\raggedleft
165\strut
\end{minipage} & \begin{minipage}[t]{0.05\columnwidth}\raggedleft
3693\strut
\end{minipage} & \begin{minipage}[t]{0.10\columnwidth}\raggedleft
11.5\strut
\end{minipage} & \begin{minipage}[t]{0.08\columnwidth}\raggedleft
70\strut
\end{minipage} & \begin{minipage}[t]{0.05\columnwidth}\raggedleft
1\strut
\end{minipage} & \begin{minipage}[t]{0.19\columnwidth}\raggedright
buick skylark 320\strut
\end{minipage}\tabularnewline
\begin{minipage}[t]{0.03\columnwidth}\raggedleft
18\strut
\end{minipage} & \begin{minipage}[t]{0.07\columnwidth}\raggedleft
8\strut
\end{minipage} & \begin{minipage}[t]{0.10\columnwidth}\raggedleft
318\strut
\end{minipage} & \begin{minipage}[t]{0.08\columnwidth}\raggedleft
150\strut
\end{minipage} & \begin{minipage}[t]{0.05\columnwidth}\raggedleft
3436\strut
\end{minipage} & \begin{minipage}[t]{0.10\columnwidth}\raggedleft
11.0\strut
\end{minipage} & \begin{minipage}[t]{0.08\columnwidth}\raggedleft
70\strut
\end{minipage} & \begin{minipage}[t]{0.05\columnwidth}\raggedleft
1\strut
\end{minipage} & \begin{minipage}[t]{0.19\columnwidth}\raggedright
chrysler satellite\strut
\end{minipage}\tabularnewline
\begin{minipage}[t]{0.03\columnwidth}\raggedleft
16\strut
\end{minipage} & \begin{minipage}[t]{0.07\columnwidth}\raggedleft
8\strut
\end{minipage} & \begin{minipage}[t]{0.10\columnwidth}\raggedleft
304\strut
\end{minipage} & \begin{minipage}[t]{0.08\columnwidth}\raggedleft
150\strut
\end{minipage} & \begin{minipage}[t]{0.05\columnwidth}\raggedleft
3433\strut
\end{minipage} & \begin{minipage}[t]{0.10\columnwidth}\raggedleft
12.0\strut
\end{minipage} & \begin{minipage}[t]{0.08\columnwidth}\raggedleft
70\strut
\end{minipage} & \begin{minipage}[t]{0.05\columnwidth}\raggedleft
1\strut
\end{minipage} & \begin{minipage}[t]{0.19\columnwidth}\raggedright
chrysler rebel sst\strut
\end{minipage}\tabularnewline
\begin{minipage}[t]{0.03\columnwidth}\raggedleft
17\strut
\end{minipage} & \begin{minipage}[t]{0.07\columnwidth}\raggedleft
8\strut
\end{minipage} & \begin{minipage}[t]{0.10\columnwidth}\raggedleft
302\strut
\end{minipage} & \begin{minipage}[t]{0.08\columnwidth}\raggedleft
140\strut
\end{minipage} & \begin{minipage}[t]{0.05\columnwidth}\raggedleft
3449\strut
\end{minipage} & \begin{minipage}[t]{0.10\columnwidth}\raggedleft
10.5\strut
\end{minipage} & \begin{minipage}[t]{0.08\columnwidth}\raggedleft
70\strut
\end{minipage} & \begin{minipage}[t]{0.05\columnwidth}\raggedleft
1\strut
\end{minipage} & \begin{minipage}[t]{0.19\columnwidth}\raggedright
ford torino\strut
\end{minipage}\tabularnewline
\begin{minipage}[t]{0.03\columnwidth}\raggedleft
15\strut
\end{minipage} & \begin{minipage}[t]{0.07\columnwidth}\raggedleft
8\strut
\end{minipage} & \begin{minipage}[t]{0.10\columnwidth}\raggedleft
429\strut
\end{minipage} & \begin{minipage}[t]{0.08\columnwidth}\raggedleft
198\strut
\end{minipage} & \begin{minipage}[t]{0.05\columnwidth}\raggedleft
4341\strut
\end{minipage} & \begin{minipage}[t]{0.10\columnwidth}\raggedleft
10.0\strut
\end{minipage} & \begin{minipage}[t]{0.08\columnwidth}\raggedleft
70\strut
\end{minipage} & \begin{minipage}[t]{0.05\columnwidth}\raggedleft
1\strut
\end{minipage} & \begin{minipage}[t]{0.19\columnwidth}\raggedright
ford galaxie 500\strut
\end{minipage}\tabularnewline
\bottomrule
\end{longtable}

\hypertarget{podmuxednky-a-body}{%
\subsection{Podmínky a body}\label{podmuxednky-a-body}}

Úkol i protokol vypracujte samostatně. Pokud na řešení nějaké úlohy
budete přesto s někým spolupracovat, radit se, nezapomeňte to u odpovědi
na danou otázku uvést. Tato zápočtová úloha obsahuje 10 otázek po 1
bodu. Celkem za 3 zápočtové úlohy bude možné získat 30 bodů, přičemž pro
získání zápočtu je potřeba více jak 20 bodů. Další dodatečné body mohu
případně individuálně udělit za řešení mini domácích úkolů z
jednotlivých hodin.

\hypertarget{odevzduxe1nuxed}{%
\subsection{Odevzdání}\label{odevzduxe1nuxed}}

Protokol ve formátu pdf odevzdejte prostřednictvím MS Teams, nejpozději
do 10. 11. 2021.

\hypertarget{pux159edzpracovuxe1nuxed-dat}{%
\section{Předzpracování dat:}\label{pux159edzpracovuxe1nuxed-dat}}

\hypertarget{otuxe1zka-01}{%
\subsection{Otázka 01}\label{otuxe1zka-01}}

Zjistěte, zdali data neobsahují chybějící hodnoty (\texttt{NA}). Pokud
ano, tak rozhodněte zdali můžete příslušná pozorování z dat odstranit a
proč. Které proměnné jsou kvantitativní a které kvalitativní? Jeli možno
některé zařadit do obou skupin, pro kterou se rozhodnete? Které proměnné
budete brát jako faktorové a proč? Spočtěte základní statistiky pro
jednotlivé proměnné.

\hypertarget{otuxe1zka-02}{%
\subsection{Otázka 02}\label{otuxe1zka-02}}

Proměnnou \texttt{mpg} nahraďte proměnnou \texttt{spotreba} kde bude
místo počtu ujetých mil na galon paliva uvedena hodnota počet litrů na
100 Km. Proměnnou \texttt{cylinders} přejmenujte na
\texttt{pocet\_valcu}. Proměnnou \texttt{displacement} přejmenujte na
\texttt{zdvihovy\_objem} a převeďte z kubických palců na litry.
Proměnnou \texttt{horsepower} přejmenujte na \texttt{výkon} a převeďte
na kW. Proměnnou \texttt{weight} přejmenujte na \texttt{hmotnost} a
převeďte z liber na kilogramy. Odstraňte proměnnou
\texttt{acceleration}. Proměnnou \texttt{model.year} přejmenujte na
\texttt{rok\_vyroby} a upravte ji tak, aby její hodnoty popisovaly celý
rok 19XX. Proměnnou \texttt{origin} přejmenujte na \texttt{puvod} a
upravte ji tak, že místo 1 bude USA, místo 2 EUR a místo 3 JAP. Z
proměnné \texttt{car.name} vytvořte proměnnou \texttt{vyrobce} podle
prvního slova obsaženého v řetězci proměnné \texttt{car.name}.

\hypertarget{vizualizace-dat}{%
\section{Vizualizace dat}\label{vizualizace-dat}}

\hypertarget{otuxe1zka-03}{%
\subsection{Otázka 03}\label{otuxe1zka-03}}

Vykreslete scatterploty pro všechny numerické proměnné. Pro proměnné
\texttt{spotreba} a \texttt{hmotnost} vykreslete histogramy spolu s
jádrovými odhady hustot. Pro proměnné \texttt{pocet\_valcu} a
\texttt{rok\_vyroby} vykreslete krabicové diagramy, kde odezvou bude
\texttt{spotreba}. Je z těchto grafů vidět, že některá auta mají jinou,
než očekávanou spotřebu? Navrhněte úpravu těchto dvou proměných
(případně úpravu datasetu) tak, aby obě proměnné \texttt{pocet\_valcu} a
\texttt{rok\_vyroby} byly faktorové a obsahovaly právě 3 úrovně. Pro
takto upravená data vykreslete místo výše zmííněných boxplotů violin
ploty.

\hypertarget{otuxe1zka-04}{%
\subsection{Otázka 04}\label{otuxe1zka-04}}

Pro kombinace faktorizovaných proměnných \texttt{pocet\_valcu},
\texttt{rok\_vyroby} a \texttt{puvod} vykreslete spotřebu aut, aby bylo
na obrázku vidět, jestli se liší spotřeba u aut pocházejících z různých
kontinentů v závislosti na počtu válců, roku výroby a naopak. Zobrazte
jen kombinace s relevantním počtem dat.

\hypertarget{otuxe1zka-05}{%
\subsection{Otázka 05}\label{otuxe1zka-05}}

Pro auta výrobce Chrysler vykreslete závislost spotřeby na váze
automobilu, kde jednotlivé události označíte barvou podle počtu válců a
velikost bodů v grafu bude odpovídat objemu motoru.

\hypertarget{jednoduchuxfd-lineuxe1rnuxed-model}{%
\section{Jednoduchý lineární
model}\label{jednoduchuxfd-lineuxe1rnuxed-model}}

\hypertarget{otuxe1zka-06}{%
\subsection{Otázka 06}\label{otuxe1zka-06}}

Sestavte jednoduchý regresní model (s i bez interceptu), kde
vysvětlovaná proměnná bude spotřeba automobilu. Spočtěte pro oba modely
\(R^2\) a \(F\) statistiky, co nám o modelech říkají. Vyberte jeden z
nich a zdůvodněte proč ho preferujete. Na základě zvoleného modelu
zjistěte, zdali spotřeba automobilu závisí na hmotnosti automobilu.
Pokud ano, o kolik se změní očekávaná spotřeba automobilu pokud se jeho
hmotnost zvýší o 1000kg?

\hypertarget{otuxe1zka-07}{%
\subsection{Otázka 07}\label{otuxe1zka-07}}

Sestavte obdobný model jako v předchozí otázce, ale pouze na základě dat
výrobce Chrysler. Liší se tento model od předchozího? Jaký model
vykazuje silnější linearní vztah mezi hmotností a spotřebou a proč? O
kolik roste spotřeba s rostoucí hmotností pro vozy Chrysler rychleji než
pro libovolný automobil? Spočtěte 95\% konfidenční intervaly pro
regresní koeficienty popisující sklon regresnní přímky v obou modelech a
zjistěte, zdali se protínají? Co z toho můžeme vyvozovat? Na základě
těchto modelů zjistěte o kolik procent bude mít automobil značky
Chrysler a hmotnosti 1,5 tuny vyšší očekávanou spotřebu než průměrný
automobil o stejné hmotnosti.

\hypertarget{otuxe1zka-08}{%
\subsection{Otázka 08}\label{otuxe1zka-08}}

Vykreslete scatterplot hmotností automobilů a jejich spotřeby. Do tohoto
grafu vykreslete regresní přímku modelu s interceptem i bez. Sestrojte
navíc lineární model, kde budete uvažovat, že spotřeba závisí na
kvadrátu hmotnosti. Příslušnou křivku popisující odhady středních hodnot
z tohoto modelu přidejte do obrázku k oboum předchozím modelům. Pro
účely predikce spotřeby automobilů, na základě jakých statistik byste
mezi těmito modely vybírali, nebo byste se rozhodovali na základě něčeho
jiného a proč?

\hypertarget{otuxe1zka-09}{%
\subsection{Otázka 09}\label{otuxe1zka-09}}

Pro vámi vybraný finální lineární model popisující vztah mezi hmotností
a spotřebou automobilu ověřte předpoklady pro použití metody nejmenších
čtverců. Každý předpoklad zmiňte a uveďte jak byste ho validovali pomocí
reziduí.

\hypertarget{otuxe1zka-10}{%
\subsection{Otázka 10}\label{otuxe1zka-10}}

Přidejte k vysvětlující proměné \texttt{hmotnost}, i proměnnou
\texttt{puvod}. Navrhněte aditivní lineární model (případně 3 modely pro
každý region zvlášť), ve scatterplotu vykreslete 3 skupiny různými
barvami a data proložte třemi odpovídajícími regresními přímkami.
Uvažujeme 3 auta o hmotnosti 2 tuny zastupující jednotlivé regiony
původu. Sestrojte 90\% konfidenční intervaly okolo očekávaných spotřeb a
na jejich základě rozhodněte, zdali a jak se očekávané spotřeby budou
lišit. Je to porovnávání správné? Zdůvoněte.

\end{document}
